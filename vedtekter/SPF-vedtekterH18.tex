\documentclass[11pt,norsk,a4paper]{article}
\usepackage[norsk]{babel}
\usepackage[utf8]{inputenc}
\usepackage{cleveref,parskip,textcomp,fullpage,fancyhdr,graphicx,lastpage}
\usepackage[margin=2cm,headsep=1cm]{geometry}
\usepackage[bitstream-charter]{mathdesign}
\usepackage[T1]{fontenc}
\usepackage{enumitem}

% Fractions
\usepackage{xfrac}
\DeclareInstance{xfrac}{default}{text}%
{%
	scale-factor = 1.2,
	numerator-bot-sep = 3pt,
	denominator-bot-sep = -1pt,
}

% Redefines sections and subsections in the format §1-1
\renewcommand{\thesection}{§ \arabic{section}}
\renewcommand{\thesubsection}{\thesection-\arabic{subsection}}
% TODO: Skriv inn gyldig dato her
\newcommand{\thedate}{23.08.2018}

% Headers
\pagestyle{fancy}
\fancyhf{}
\fancyhead[L]{\textit{Vedtekter for Studenkjellernes personalforening av \thedate}}
\fancyhead[R]{\thepage\ av\ \pageref{LastPage}}

% Redefines first level of enumerate in the format a), b), ...
\renewcommand{\theenumi}{\alph{enumi}}
\renewcommand{\labelenumi}{\theenumi)} % chktex 9 chktex 10

\author{}
\title{%
    {\LARGE\textbf{Vedtekter for Studenkjellernes personalforening}}\\%
	{\small oppdatert av generalforsamling \thedate}%
}
\date{}

\begin{document}
\maketitle{}

\section{Navn\label{sec:navn}}
Foreningens navn er \emph{Studentkjellernes personalforening}.


\section{Formål\label{sec:formål}}
Foreningens formål er å administrere personaltjenester for medlemmer av 
\emph{Studentkjellernes personalforening}.
Med personaltjenester menes lønnet arbeid utført i regi av medlemsforeningene.


\section{Medlemskap\label{sec:3}}

\subsection{Mulige medlemmer}
Alle foreninger ved Universitet i Oslo som  har egen bardrift, eller driver med utleievirksomhet, kan bli medlemmer.

\subsection{Opptak}
Nye medlemmer tas opp etter behandling av skriftlig forespørsel til styret.

\subsection{Medlemsavtale}
Alle medlemmer har en egen medlemsavtale som skal regulere ansvarsfordelingen
mellom Studentkjellernes personalforening og hver enkel medlemsforening.
Avtalen nyforhandles hvert høstsemester og skal være på plass senest
fire uker etter generalforsamling. 
Brudd på medlemsavtalen eller forsømmelse av avtaleforhandling
kan føre til suspensjon eller oppsigelse av medlemskapet.

\subsection{Medlemsrepresentasjon\label{sec:medlemsrepresentanter}}
Alle medlemsforeninger har et medlem og et varamedlem i styret.
Styrerepresetanten og varamedlemmet velges av medlemsforeningen.


\section{Generalforsamling}

\subsection{Ordinær generalforsamling}
\begin{enumerate}
	\item Ordinær generalforsamling innkalles med tre ukers varsel. 
	\item Ordinær generalforsamling holdes to ganger i året: for vårsemesteret i januar eller februar, og for høstsemesteret i august eller september.
	\item Generalforsamlingen skal behandle styrets beretning, regnskap og budsjett. 
	\item Generalforsamlingen skal godkjenne de styremedlemmer medlemsforeningen 
		har oppnevnt til styret, jf.~\ref{item:valgbar}, 
		og velge leder for Studentkjellernes personalforening blant disse og eventuelle andre medlemsrepresentanter.
\end{enumerate}

\subsection{Ekstraordinær generalforsamling}
\begin{enumerate}
	\item Ekstraordinær generalforsamling innkalles med to ukers varsel. 
	\item Styret eller to av medlemsforeningene kan innkalle til ekstraordinær generalforsamling.
\end{enumerate}

\subsection{Stemme- og møterett}
\begin{enumerate}
	\item En representant for hver av medlemsforeningene har stemmerett på generalforsamlingen.
	\item Styret i foreningen, styremedlemmer i medlemsforeningene,
samt de som har mottatt honorar de siste 24 månedene fra foreningen, har møterett på
generalforsamlingen.
\end{enumerate}

\subsection{Stemmeberettighet}
Generalforsamlingen er stemmeberettiget når \sfrac{3}{4} av medlemmene er tilstede.

\section{Styret}
\subsection{Hovedoppgave}
Styrets hovedoppgave er å forvalte medlemsforeningenes behov for personaltjenester.
De skal ta opp nye medlemmer, inngå medlemsavtaler og følge disse.

\subsection{Sammensetning\label{sec:sammensetning}}
\begin{enumerate}[ref=\thesubsection~punkt \theenumi]
	\item Styret er sammensatt av en representant fra hver medlemsforening. Hver representant har en vararepresentant, jf.~\ref{sec:medlemsrepresentanter}. Samtlige medlemsforeningsrepresentanter har møterett.
	\item For å være valgbar til styret må en styrerepresentant være aktiv frivillig i sin egen forening ved starten av vervet, og godkjent av sin egen forenings hovedstyre\label{item:valgbar}.
	\item Dersom et styre- eller varamedlem fratrer må medlemsforeningen velge en ny representant innen en uke\label{item:styre-fratrer}.
\end{enumerate}

\subsection{Verv}
\begin{enumerate}
	\item Faste verv i styret er styreleder, nestleder, økonomiansvarlig og sekretær\label{item:faste-verv}. 
	\item Faste verv jf. \cref{item:faste-verv} velges på generalforsamling og har en varighet fram til neste ordinære generalforsamling.
	\item Så langt det er mulig skal vervet som styreleder gå på omgang blant medlemsforeningene. 
	\item Styreleder er daglig leder for foreningen.
	\item Dersom et styremedlem fratrer fra et fast verv, adskilt eller i helhet jf.~\ref{item:styre-fratrer}, må det arrangeres ekstraordinær generalforsamling for å fylle vervet fram til neste ordinære generalforsamling.
\end{enumerate}


\section{Protokolltilførsel}
Styret skal føre og godkjenne protokoll fra alle sine møter.
Kopi av protokoll skal sendes til driftslederne i alle medlemsforeningene.


\section{Økonomi}
Foreningen skal ikke ha som mål å tjene penger og skal derfor ikke gå med overskudd.
Foreningen har mulighet til å opparbeide seg et fond.


\section{Signatur}
Daglig leder har signaturrett.


\section{Vedtekter\label{sec:vedtekter}}
\begin{enumerate}
	\item Vedtektene kan endres på ordinær og ekstraordinær generalforsamling.
	\item Vedtekter må ha minst \sfrac{3}{4} flertall.
	\item Medlemsforeningenes hovedstyrer er høringsorgan for vedtektsendringer.
	\item Vedtektsendringer skal være klare og sendes ut til medlemsforeningene senest to uker før generalforsamling.
\end{enumerate}


\section{Oppløsning\label{sec:opplosing}}
\begin{enumerate}
	\item Vedtak om oppløsning må være enstemmig.
	\item Foreningen kan oppløses ved vedtak om oppløsning på to etterfølgende generalforsamlinger, med minimum én måned mellomrom.
	\item Foreningens gjenværende eiendeler etter gjeldsoppgjør skal fordeles blant medlemsforeningene etter enighet på siste oppløsningsgeneralforsamling. Dersom enighet ikke oppnås tilfaller eiendeler til SiO Foreninger.
\end{enumerate}

\end{document} % chktex 17
