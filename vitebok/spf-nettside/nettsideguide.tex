\documentclass[12pt,norsk]{article}
\usepackage[utf8]{inputenc}
\usepackage[norsk]{babel}
\usepackage{lmodern}
\usepackage{hyperref}

\begin{document}

\title{SPF-lønnsubetalingguide}
\author{Alexander Fleischer}
\maketitle

\section{SPF-nettsiden}
\label{spf}

Det er superenkelt å legge inn både personell og arrangementer
på SPF-nettsiden. Denne seksjonen gir deg en barnevennlig
innføring.\\

\noindent \textbf{Lenke:} \url{http://spf.lantea.net}.
Aleksi 
Luukkonen\footnote{\href{mailto:aleksiml@ifi.uio.no}{aleksiml@ifi.uio.no}} 
er administrator.\\

\noindent Passord og brukernavn er spesielt for hver forening.
Spør Aleksi hvis dere ikke har fått laget bruker ennå.

\subsection{Hvordan legge inn folk}
\label{spflonn}

Trykk på «Workers». 
Under «Add new worker», skriv inn
\begin{itemize}
  \item Navn («Name»)
  \item Folkeregistrert adresse («Address»)
  \item Bankkontonummer («Account no»)
  \item Personnummer («Person id»)
  \item Norlønn-nummer\footnote{Merk: legges inn etter
    at personen er lagt inn i Norlønn. Du kan fint
    legge inn personell uten det.}
    («Norlonn number»)
\end{itemize}
og trykk «Save».\\

\noindent Under «List of workers» ser man alt aktivt personell.
Ved å trykke på et navn, kan man endre lønnsinformasjonen
til personen.
For å gjøre personen inaktiv, avhuker man «Active»-boksen.
Når en person er inaktiv, dukker den ikke opp når man
legger inn arrangementer (se seksjon \ref{spfarr}).\\

\noindent Det går foreløpig ikke an å fjerne 
personell direkte fra nettsiden.
Se seksjon \ref{annet} for å slette personell.

\subsection{Hvordan legge inn arrangementer}
\label{spfarr}

Trykk på «Events» og så «Add new». 
Velg ansatt, og skriv inn
\begin{itemize}
  \item Arrangementnavn («Name»)
  \item Dato («Date»)
  \item Nettolønn\footnote{Lønn uten noen avgifter 
    (typisk mva. og SPF-avgift)} («Wage»)
  \item Arbeidstimer («Hours»)
\end{itemize}

deretter trykker du «Submit». 
Arrangementet vil da dukke opp under «Not processed yet».\\

\noindent Arrangementer som 
allerede er lagt inn, kan foreløpig ikke slettes 
eller endres direkte fra nettsiden.
Det er heller ikke mulig å legge inn flere enn syv ansatte
per arrangement.
Se seksjon \ref{annet} for sletting, endring og for å legge inn
flere enn syv ansatte.
\subsection{Andre funksjoner}
\label{annet}

Oppe til høyre på nettsiden er det en lenke som heter «Admin panel». 
Hvis du har fått en form for adminprivilegier
\footnote{Ta kontakt med Aleksi.},
vil du kunne slette personell og arrangementer gjennom adminpanelet.
Når du lager / endrer gjennom admin\-panelet,
kan du legge til så mange ansatte du måtte ønske.

\section{Norlønn -- denne seksjonen blir oppdatert etter hvert}
Gjennomgang av lønnsutbetaling
skjer \textbf{tirsdag 3. mars kl. 15.30} på Uglebo.\\

\noindent \textbf{Lenke:} \url{http://norlonn.no}\\
\textbf{Brukernavn:} spfstyret\\
\textbf{Passord:} curaCrapula2015

%\subsection{Hvordan legge inn folk}
%De som betaler ut lønn legger inn folk i Norlønn.
%Dette må gjøres manuelt (enn så lenge).
\end{document}
