\documentclass[a4paper,norsk]{article}
\usepackage[T1]{fontenc,url}
\usepackage[utf8]{inputenc}

\title{SPFs generalforsamling V2017}
\author{Grupperom Awk, Ole Johan Dahls hus}

\begin{document}
\maketitle

\textbf{Oppmøte:} 
    Maximiliano Horta (RF-regi), Susanne Tande (Realistforeningen), Thor Høgås (Cybernetisk Selskab), 
    Oskar Dønnum (Samfunnsvitenskapelig Fakultetsforening), 
    Sole Lindvåg Lie (Medicinerforeningen),
    Hanna Elisabeth Fjugstad Hansen (Kjellerutvalget), Erling Thorsen (Filologisk Forening).

\textbf{Ikke til stede:} Lea Marie Tallerud (Samfunnsvitenskapelig Fakultetsforening).


\section{Godkjenning av innkalling}
Den var sendt ut i god tid før i dag, omtrent tre uker tidligere.

\section{Valg av ordstyrer og referent}
Maximiliano blir ordstyrer og Susanne blir referent.

\section{Hei og velkommen til nye fjes}
Sole og Hanna er nye. Hei til dere. Usikker på om Tora fortsatt er representant fra FF. \\
\\
Max har frem til nå vært ansvarlig for å betale ut lønn. Susanne har vært sekretær og har skrevet referater og hentet post. Thor har vært leder, og hatt overordnig ansvar. Oskar har vært ansvarlig for økonomien. Lea var nestleder og ansvarlig for innkallinger og nettsiden. 
  
\section{Valg av styreroller}
\subsection*{Leder}
\textbf{Arbeidsoppgaver:}
\begin{itemize}
	\item Signeringsrett i Altinn
	\item Hente skattekort
	\item Levere inn a-melding til skatteetaten hver måned
	\item Oversikt over papirer
	\item Ting som må løses fortløpende
	\item Vervet skal gå på rundgang
\end{itemize}
\textbf{Leder vår 2017:} Thor Høgås.

\subsection*{Nestleder}
Arbeidsoppgaver:
\begin{itemize}
	\item Kalle inn til møter
	\item Holde epostlisten oppdatert
\end{itemize}
\textbf{Nestleder vår 2017:} Lea Tallerud

\subsection*{Økonomiansvarlig}
Arbeidsoppgaver:
\begin{itemize}
	\item Ansvar for økonomien
\end{itemize}
Oskar har ikke egentlig tid til å fortsette.\\
\textbf{Økonomiansvarlig vår 2017:} Hanna Hansen

\subsection*{Sekretær}
Arbeidsoppgaver:
\begin{itemize}
	\item Skrive referater fra møtene
	\item Hente post i postboksen
\end{itemize}
\textbf{Sekretær vår 2017:} Susanne Tande
\subsection*{Øvrige styremedlemmer}
\textbf{Øvrige styremedlemmer:} Erling, Sole og Max.

\section{Rolledelegering}
Vi har to roller:
\begin{enumerate}
	\item Lukke periode og sende ut faktura
	\item Betale ut lønn
\end{enumerate}
Erling og Sole bytter på å lukke perioden. De kommuniserer seg i mellom hver måned om hvem som gjør det. Maximiliano betaler ut lønn.

\section{Status}
Vi hadde 112 registrerte utlån forrige semester.

\subsection*{Økonomi}
Vi ligger OK an, økonomisk sett. Ingen kriselån eller liknende er utdelt. Mesteparten fra i fjor er ferdigført, men vi mangler noe bilagsføring. \\
\\
SPF har vært i minus tidligere, og hatt lån hos kjellerforeningene. Vi satte da opp SPF-avgiften til 30\%, og har i år gått i pluss. \\
\\
Har fått opprettet konto hos DNB. Har integrering med Visma. Må fikse slik at vi får tilgang uten å bruke Thors bankbrikke. \\
\\
Lønn fra januar har ikke blitt betalt ut ennå.

\subsection*{SPF-siden}
Den er oppdatert og ser veldig bra ut. Spesielt sidene om fakturering og lønn har blitt mer oversiktlige. En lekker fremdriftsindikator har blitt lagt inn, så man enkelt får oversikt over når ting burde skje.

\section{Eventuelt}
\subsection*{Gå over til DNB-konto}
Burde gå over til å bruke DNB-kontoen (bruker for øyeblikket Nordea). Burde skaffe to ekstra brukere til denne, slik at både leder, økonominasvarlig og personen ansvarlig for utbetaling av lønn har tilgang. Det koster 40 kr/mnd per ekstra bruker. 
\textbf{Vedtar å endre driftskonto til DNB og opprette to ekstra brukere.}

\subsection*{SPF-repo på github}
Det kan være nyttig å ha en egen plass til ting. For øyeblikket ligger det på tidligere sekretær sin personlige bruker. Kan legge hele SPF-siden der, og referater etc. 

\end{document}