\documentclass{article}[12pt]
\usepackage[utf8]{inputenc}
\usepackage[norsk]{babel}
\usepackage{lmodern}
\renewcommand{\thesubsection}{\thesection\alph{subsection}}

\begin{document}

\title{Møte mellom SiO og SPF -- møtereferat}
\date{10. mars 2015}
\maketitle
\textbf{Tilstede: }
Mike fra SiO, SPF-representant MF, SPF-representant RF, SPF-representant Uglebo, 
styreleder Uglebo, nestleder UV, SPF-representant UV, økonomiansvarlig SvFF, 
SPF-representant SvFF.

\section{Økonomisk situasjon i SPF}
\noindent Det er 175 472,33 kr i utestående lønn helt tilbake fra 2008.\\

\noindent UV har en SPF-konto der penger fra utleier blir satt. 
Den er på \~100 000 kr.
Må sjekke om disse pengene utelukkende er utestående penger til SPF,
eller om det er snakk om inntekter i form av slitasje-/admingebyr o.l.\\

\noindent Skal foreningene betale en prosentandel av det utestående beløpet
for å dekke inn lønnen? Siden vi sparer inn litt penger på
å si opp avtalen med Nitschke, kan det hende SPF-avgiften gir oss
litt kapital vi kan bruke til å betale ut lønn fra tidligere fremover.\\

\noindent I fjor ble det gitt ut et «kriselån» til SPF.
Skal man prioritere
å betale tilbake lånet, eller å betale ut gammel lønn?\\

\noindent Foreningene bør prøve å se på hvor mye penger man har fått inn for 
utleier (til lønn), og sammenligne tallene med hvor mye som har 
blitt betalt til SPF\. Minst tilbake til 2013.\\

\noindent Med utgifter til lønnsutbetalinger og størrelsen på SPF-avgiften,
er det kanskje usannsynlig at hele kriselånet blir betalt tilbake
til foreningene. Iallfall i løpet av 2015/2016.\\

\section{Hva bør man gjøre for å ikke komme i samme situasjon igjen?}

\noindent SPF bør ikke betale ut lønn før penger 
fra foreningene har kommet inn på konto.\\

\noindent Styret i hver forening bør følge opp, og få oppdateringer fra
sin SPF-representant jevnlig for å sørge for at jobben blir gjort.\\

\noindent Det bør være i kjellernes interesse at SPF går knirkefritt.
Foreningene må tenke at SPF er «sin» organisasjon.\\

\noindent Hvordan statusen i SPF er bør evalueres etter et halvt år.\\

\noindent Sørge for at fakturaer blir sendt ut, lønn blir utbetalt osv.\\

\noindent Prioritering fremover: betale ut lønn fra 2014 og 2013, deretter
se på kriselånet. Forhåpentligvis samles det opp litt egenkapital
gjennom SPF-avgiften (30 \%) som kan brukes til å betale dette.

\end{document}
