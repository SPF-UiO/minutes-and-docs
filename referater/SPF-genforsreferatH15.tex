\documentclass{article}[12pt]
\usepackage[utf8]{inputenc}
\usepackage{parskip}
\renewcommand{\thesubsection}{\thesection \alph{subsection}}

\begin{document}

\title{Generalforsamling i Studentkjellerenes personalforening -- referat}
\date{11.09.2015}
\maketitle

\textbf{Oppmøte:} (RF-regi), Realistforeningen, Cybernetisk Selskap, Samfunnsvitenskapelig Fakultetsforening, Filologisk Forening, Kjellerutvalget

\section{Valg av ordstyrer og referent}

\textbf{Ordstyrer:} Jan Kristian Furulund, godkjent.

\textbf{Referent:} Alexander Fleischer, godkjent.

\section{Godkjenning av innkallelse}
	- Innkallelse er godkjent, men den bør komme tidlig nok til at man får mulighet til å sende inn endringsforslag

\section{Godkjenning av dagsorden}
	- Dagsorden er godkjent. Legge til punkt om valg av sekretær (hvis endringsforslag går igjennom).
	
\section{Fremleggelse av leders semesterberetning}
	- Semesterberetning er fremlagt i papirform.

\section{Fremleggelse av økonomi}
	- SPF har tilbakebetalt lønn fra tidligere år. Tilbake til 2013, og tidligere
	ved forespørsel fra arbeidstaker. 
	Dette er fordi vi ikke har fullstendige lønnsopplysninger 
	om alle som har jobbet tidligere. 
	Utesåtende lønn 2008-2013 var ved starten av
	året rundt 225 000,–.
	
	- Underskudd er pga. tilbakebetaling av tidligere lønn, men SPF-avgiften er
	stor nok til at vi går i overskudd over tid.
	
	- Det er tilbakebetalt utestående penger fra Kjellerutvalget.
	
	- SvFF, MF, RF-regi og RF må gi oversikt over mulig utestående beløper fra
	tidligere år.
	
	- Uglebo betaler tilbake utestående beløp over flere regnskapsperioder fremover.
	
	- Kommentar: høy negativ egenkapital grunnet utestående lønn. 
	Få unna lønnskrav fra tidligere så det ikke forstyrrer planen fremover.
	Tallene vi har nå er ikke reelle når man tar vekk beløp som (sannsynligvis) 
	ikke kommer til å bli
	betalt ut.
	
	- Kommentar: har vi budsjett? Nei, men vi har en kalkyle på SPF-avgiften som
	gjør at alle faste kostnader er tatt høyde for.  

\section{Statuttendringsforslag}
	Vi er fem stemmeberettigede, seks foreninger (usikkerhet til RF-regi) 
	med stemmerett sitter i SPF.
	\begin{enumerate}
	\item \textbf{Formål}. Ingen innspill. Enstemmig vedtatt.
	\item \textbf{Medlemskap}. Innspill: ligger den gamle avtalen til grunn og man reforhandler mtp. endringer?
	 Ja, det er meningen med forslaget. Enstemmig vedtatt.
	 \item \textbf{Endre frister.} Rar ordlyd. Endring av §4.1 enstemmig vedtatt.
	 Endring av §9 til 'ti dager før' enstemmig vedtatt. 
	 Tillegget ble enstemmig nedstemt. 
	 Kommentar: Det bør komme et bedre forslag 
	 som rydder opp i dette til neste generalforsamling.
	 \item \textbf{Generalforsamling, 4.1.} Innspill: rar ordlyd. Enstemmig nedstemt.
	 Kommentar: Generalforsamling anbefaler at man legger frem budsjett i september og
	 reviderer i februar.
	 \item \textbf{Generalforsamling, 4.2.} Innspill: styret burde kunne kalle inn til ekstraordinær generalforsamling siden det er de som (bør) ha kontroll på sakene.
	 Enstemmig nedstemt.
	 \item \textbf{Generalforsamling, 4.4.} 
	 Innspill: gjøre klart at $ceiling(0.75\times 6)=5$.
	 Enstemmig nedstemt.
	 \item \textbf{Styret, 5.4.} Innspill: meningen med forslaget er å sørge for
	 at 6 av 7 foreninger er tilstedet når styremedlemmene velges.
	 Enstemmig vedtatt.
	 Kommentar: man kan senere se på et endringsforslag der man legger til
	 at styret kan supplere styremedlemmer til en av de rollene hvis innehaveren trekker seg.
	 \item \textbf{Styret, 5.5.} Mening: gjøre det enklere for folk som er interesserte
	 i SPF å gjøre en innsats der, uten å måtte ha et større verv. 
	 Gir også foreningene større frihet til å velge SPF-representant.
	 Innspill: bør man ikke sitte i styret så man har kontroll på hva som skjer i foreningen
	 sin og kan berette om hva SPF gjør til styret sitt?
	 Enstemmig vedtatt.
	 \item \textbf{Styret, 5.4.} Mening: at det er økonomisk kompetanse i SPF.
	 Innspill: er det bedre med økonomikurs for store foreninger? SPF styrer ikke med mva.
	 Bør man pålegge hele styret dette?
	 Nedstemt med 3 mot, 1 for og 1 avholdende.
	 \item \textbf{Styret, 5.6.} Ingen innspill. Enstemmig vedtatt.
	 \item \textbf{Protokolltilførsel.} Enstemmig vedtatt.
	 \item \textbf{Vedtekter.} Enstemmig nedstemt.
\end{enumerate}

\section{Valg av nytt styre}
   \subsection{Leder}
   - Henrik stiller. Velges ved akklamasjon.
   \subsection{Nestleder}
   - Snorre stiller. Satt som nestleder forrige semester. Snorre velges ved akklamasjon.
   \subsection{Økonomiansvarlig}
   - Christiane stiller. Var leder forrige semester. Stilte til leder fordi hun
   ville styre med økonomien. Nå som det er eget verv passer det bedre. Christiane velges ved akklamasjon.
   \subsection{Sekretær}
   - Alexander stiller. Var sekretær forrige semester. Alexander velges ved akklamasjon.
   \subsection{Øvrige styremedlemmer}
   - Christian (og ev. Geir og Emma) 
   er eneste gjenværende medlemmer og velges ved akklamasjon.
   
\section{Eventuelt}
\subsection{Signering av referat og informasjon om nye medlemmer}
Skjer rett etter generalforsamlingen. Én fra hver forening. 
Nye styremedlemmer må skrive ned fullt navn og e-postadresse. 
\end{document}