\documentclass[a4paper,norsk]{article}
\usepackage[T1]{fontenc,url}
\usepackage[utf8]{inputenc}

\title{SPFs generalforsamling V2018}
\author{Grupperom Awk, Ole Johan Dahls hus}

% PARAGRAPH SPACING
\setlength{\parskip}{1em}

\begin{document}
\maketitle
\textbf{Oppmøte:
	\begin{itemize}
		\item[Cyb] Thor Høgås, Benjamin Breiby
		\item[MF] Myra Hegrand
		\item[FF] Magne Klasson
		\item[KU] Alina Zjalilova, Amund Asheim
		\item[RF] Susanne Tande, Jan Ole Åkerholm
		\item[SvFF] Live Funjem
		\item[Regi] Helje Svensson 
	\end{itemize}
	} 
\tableofcontents

\section{Valg av ordstyrer og referent}
\textbf{Susanne blir valgt til referent, Thor blir valt til  ordstyrer.}

\section{Godkjenning av innkalling}
Vi ba spesifikt om noen hadde noen innsigelser angående den korte fristen, så vi håper innkallingen kan godtas selv om den var noe sen. Saken er at vi kalte inn til en genforsamling innen fristen, men valgte å utsette denne på grunn av fravær og sykdom blant medlemmene.

\textbf{Innkalling godkjent.}

\section{Godkjenning av dagsorden}
Dagsorden godkjent.

\section{Fremleggelse av leders semesterberetning}
Semesterberetning ligger vedlagt.
\begin{itemize}
	\item Styret har møttes jevnlig annenhver uke, det har vært fint
	\item Har fått småfeil som gjorde at noen gikk uten lønn, men det har stort sett gått bra med lukking av periode og utbetaling av lønn.
	\item Thor har vært med i SPF fra 2015, men det er på tide at en annen forening tar over ledervervet. Har fortsatt lyst til å bidra i SPF med råd og hjelp, selv om han ikke er utlånsansvarlig for Cybernetisk Selskab.
	\item Et best mulig regnskap for SPF er positivt for alle medlemsforeningene. Det er logisk at utlånsansvarlig er med i SPF, men det kan være fint om økonomiansvarlige fra foreningene også får en liten innføring.
\end{itemize}

\section{Fremleggelse av økonomi}
Balanserapport og Resultatrapport ligger vedlagt.

Vi er kvitt gjelden skyldig ukjent lønn. Dette var lønn som ikke ble utbetalt for minst fem år siden, så den er avskrevet nå. For første gang på legne er SPF nogen lunde i pluss. Vi har fortsatt noe gjeld, kriselån fra RF og SvFF. Dette skal vi gradvis betale tilbake. 

Regskapsmessig går SPF lett rundt. Vi har ikke for lite penger. SPF-avgiften på 30\% er nok til å dekke arbeidsgiveravgift, yrkesskadeforsikring og pris for lønnssystemet. Det var tidligere et håp å sette ned denne, men det vil ikke gå. Siden flere kjelleren har økt timeslønnen sin, gjør avgiften også at vi får inn mer penger.

\section{Forslag til vedtektsendringer}
\subsection{Tidspunkt for Generalforsamling}
I vedtektene står det at generalforsamling skal holdes i februar og september, men vi ønsker å endre dette til "januar/februar" og "august/september", slik at man kan komme i gang tidligere med nye oppgaver i SPF.

Siden vedtektsendringsforslaget ble sendt inn etter fristen, kan vi ikke stemme over denne endringen nå. Vi kommer til å kalle inn til ekstraordinært generalforsamling for å stemme over denne vedtektsendringen. Generalforsamlingen stemmer 

\section{Minstelønn for alle medlemsforeninger} 
Det er i 2018 vedtatt av blant annet LO at alle som jobber innen overnatting-, servering- og catering-virksomhet har en minstelønn. Tidligere har foreningene valgt egen lønn. Så vidt vi forstår nå, så må vi i praksis sette opp timeslønnen  til minst 157,18 kr. 

Det er usikkert om dette vil gjelde lyd- og lysteknikere, så vi må se litt på om RF-Regi må følge disse retningslinjene.

Dersom vi velger minstelønn 160 kr, vil det med moms og spf-avgift lande på 260 kr som faktureres leietakerne. Hvis vi har minstelønn 157 kr vil det totalt faktureres 255,5 kr. 

Tariff-loven har allerede tredd i kraft. Det vil si at avtaler kjellerne har laget med en lavere lønn, men som det enda har blitt betalt for, må endres til minstelønnen. Allerede gjennomførte arrangementer trenger vi ikke å ta mer betalt for.

Dette er ikke noe å vedta, for vi har blitt pålagt å gjøre endringen. Kjellerne står selv fritt til å velge 157,18 kr eller mer. Saken er mest til orientering.


\section{Valg av nytt styre}
\subsection{Leder}
Ludvig-Johannes Carlsen fra SvFF (påtroppende utleieansvarlig) stiller til leder. 

\textbf{Ludvig velges ved akklamasjon.}

\subsection{Nestleder}
Alina fra KU stiller.

\textbf{Alina velges ved akklamasjon.}

\subsection{Sekretær}
Susanne fra RF stiller.

\textbf{Susanne velges ved akklamasjon.}

\subsection{Økonomiansvarlig}
Benjamin ønsker ikke å fortsette, han melder om lite tid og at det kommer til å bli nedprioritert. Ingen ønsker å stille til vervet. Ludvik virker som en mulig kandidat, og han kan gjøre noe av jobben som leder, men vi trenger en egen valgt person. 

Siden ingen ønsker å stille, velges ingen som økonomiansvarlig. Benjamin fungerer som interrim økonominasvarlig frem til en egen person er valgt.

\subsection{Øvrige styremedlemmer}
\textbf{Helje, Myra, Benjamin og Magne velges som øvrige styremedlemmer.}

\subsection{Rådgiver} 
Thor ønsker å fortsette som rådgiver. Egentlig kan bare representantene fra hver forening og deres varaer med i SPF-styret. Thor fungerer som vara for Benjamin. Thor ønsker likevel å kunne fortsette i SPF-styret for å bista, og fordi det er noe han syns er gøy. Han liker også å programmere på SPF-systemet. Det bryter ikke med vedektene, han vil fungere som et øvrig styremedlem uten stemmerett. 

\section{Eventuelt}
\textbf{Alina og Live velges til protokollunderskrivere.}


\end{document}