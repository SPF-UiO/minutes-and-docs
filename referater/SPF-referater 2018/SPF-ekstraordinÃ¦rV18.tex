\documentclass[a4paper,norsk]{article}
\usepackage[T1]{fontenc,url}
\usepackage[utf8]{inputenc}

\title{Ekstraordinær generalforsamling V2018}
\author{U1, Eilert Sundts hus}

% PARAGRAPH SPACING
\setlength{\parskip}{1em}

\begin{document}
\maketitle
\textbf{Oppmøte:}
	\begin{itemize}
		\item[\textbf{Cyb}] Benjamin Breiby
		\item[\textbf{MF}] Myra Hegrand
		\item[\textbf{FF}] -
		\item[\textbf{KU}] Alina Zjalilova (går tidlig), Randi Margrethe Strøm Olsen (vara)
		\item[\textbf{RF}] Susanne Tande
		\item[\textbf{SvFF}] Ludvig-Johannes Carlsen
		\item[\textbf{Regi}] - 
	\end{itemize}
\tableofcontents
\newpage


\section{Valg av ordstyrer, referent og to protokollunderskrivere}
\textbf{Susanne blir valgt til referent.}
\textbf{Ludvig-Johannes blir valgt til ordstyrer.}
\textbf{Myra og Randi blir valgt til protokollunderskrivere.}

\section{Godkjenning av innkalling}
\textbf{Godkjent.}

\section{Godkjenning av dagsorden}
\textbf{Godkjent.}

\section{Forslag til vedtektsendringer}
Vi går igjennom forslaget til vedtektsendringer som ble sendt ut, paragraf for paragraf. Stort sett enighet. Nedenfor referatføres diskusjonen som kommer opp.

Under paragraf 5-2 diskuteres det rundt valg av vara. Det nye forslaget presiserer at det fra hver forening skal velges en representant og en vara. Alle synes dette er positivt. Det diskuteres om hvorvidt det er riktig at medlemsforeningenes hovedstyre skal velge representanten og varaen, men vi kommer frem til at ordlyden er dekkende.

Under paragraf 5-3 blir det understreket at det nye forslaget (samt det gamle) ikke presiseres at det \textit{må} holdes ekstraordinær generalforsamling for å fylle verv som folk har fratrådt fra.

Under paragraf 9 har det nye forslaget gjort følgende endringer: 
\begin{itemize}
	\item Delt opp i a), b) og c)
	\item Endret fra at vedtektstendringer må være enstemmig
	\item Fjernet at vedtektsendringer må være sendt inn to uker før generalforsamling
\end{itemize}
Det påpekes at vi ikke ønsker å fjerne setningen som sier at vedtektsendringer må være sendt inn to uker før generalforsamling, og vi godtar dermed ikke denne paragrafen. Ønsker å ha et punkt d) som inneholder dette. Forøvrig er alle for de andre endringene a), b) og c). Gammel ordlyd gjelder på paragraf 9. Det kan vedtas på en senere generalforsamling.

\textbf{Samtlige paragrafer bortsett fra paragraf 9 vedtas. Paragraf 9 vedtas ikke, så gammel ordlyd gjelder.}

\section{Eventuelt}
Det har vært litt kaotisk dette året når styret møtes. Det er fint å ha lang varsel for å møte. Ønskes å ha det med i vedtektene at styret prøver å ha møter en fast dag, eller å ha en ukers varsel før møtene. 
\end{document}