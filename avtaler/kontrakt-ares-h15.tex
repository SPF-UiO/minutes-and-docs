\documentclass[a4paper,10pt]{article}

\usepackage[latin1]{inputenc}
%\usepackage{amsfonts}
\usepackage{amsmath}
%\usepackage{amssymb}
%\usepackage{amsthm}
\usepackage[a4paper, total={7.5in,10in}]{geometry}
\usepackage{hyperref}
\usepackage[norsk]{babel}
\usepackage[T1]{fontenc}
\usepackage[pdftex]{graphicx}
\usepackage[none]{hyphenat}
\usepackage{lastpage}
\usepackage{fancyhdr}

\pagestyle{fancy}
%\cfoot{\thepage\ of \pageref{LastPage}}
\author{Alexander Fleischer}
\pagenumbering{gobble}
\begin{document}
 
\section*{Ekstraregler for bruk av RF-kjelleren for Ares}

\raggedright
\begin{itemize}
    \item Denne kontrakten er bare gyldig sammen med en Låneavtale for RF-kjelleren.
\item Ares må stille med minst én ansvarlig person (heretter <<Ansvarlig>>) som setter seg 
    inn i reglene i dette dokumentet, og overser at de blir fulgt.
\item Ansvarlig må være godkjent av Realistforeningens kjellerstyre 
    (heretter <<KS>>).
\item Ares får lov til å oppholde seg i RF-kjelleren uten Realistforeningens (heretter <<RF>>) 
    tilstedeværelse, forutsatt at Ansvarlig er på stedet.
\item Ansvarlig får en nøkkel til kjelleren. Denne 
    nøkkelen er personlig, og skal ikke lånes ut til andre personer uten at KS 
    tillater dette. Hvis Ansvarlig mister nøkkelen, er den ansvarlig for
     eventuelle omkostninger for å bytte låser, samt dekke Realistforeningens tap forbundet med dette.
     Ved Låneavtalens utløp skal Ansvarlig levere nøkkelen tilbake til Realistforeningens utlånsansvarlig (heretter <<Utlånsansvarlig>>).
\item Ares får \textit{ikke} lov til å bruke inventaret som er tilgjengelig bak baren. 
    Dette inkluderer ovn, oppvaskemaskiner, glass, tallerkener, bestikk osv.
\item Mikrobølgeovn, løs vannkoker, bord, stoler, benker og krakker er 
    tilgjengelige til bruk. Ares får også lov til å benytte seg av søppelbøttene, dersom
     de de er tømt og ny pose er satt i etter avsluttet arrangement (nye poser finnes i skapet under vasken nærmest utgangsdøra).
    I tilfellet hvor det ikke finnes flere rene poser i RF-kjelleren, skal Utlånsansvarlig gis beskjed om dette
    på e-post til \url{utleiesjef@rf.uio.no} innen slutttidspunktet fastsatt i Låneavtalen.
\item Hvis lokalet ikke er ryddig og rent etter et endt arrangement, kan RF kreve
     100,-- per påbegynte kvarter for rengjøring. Hvis 
    lokalet ikke er rent eller ryddig fra et tidligere arrangement, må dette 
    rapporteres med \textit{en} gang, slik at RF får muligheten til å observere kjelleren. 
    Ares er ikke ansvarlig for å rydde opp etter annet enn egen aktivitet i 
    kjelleren. 
    
    Hvis kjelleren står uryddig før påbegynt arrangement og dette 
    ikke er rapportert, anses rotet som Ares' ansvar.
\item RFs medlemmer får lov til å oppholde seg i RF-kjelleren samtidig med Ares.
    Ares har rett til unntak fra denne regelen ved forespørsel til Utlånsansvarlig før arrangementet.
\item Ved eventuell hærverk eller skade, foruten slitasje grunnet normal bruk, er Ansvarlig ansvarlig for eventuelle omkostninger
    forbundet med dette.
\item Ares kan etter avtale benytte RF-kjelleren to søndager i løpet av avtaleperioden. Det er ingen ekstra kostnader forbundet med dette,
    men forespørselen bør komme innen tre uker før den aktuelle datoen.
\item Realistforeningen kan ved en enkeltanledning benytte kjelleren på en dag beskrevet i Låneavtalen, så fremt det avtales med Ares
    innen tre uker før den aktuelle datoen. Deretter avtaler Utlånsansvarlig og Ansvarlig en ny dato der Ares kan bruke kjelleren som
    erstatning.
\item Ved brudd på reglene nevnt i dette dokumentet, samt i Låneavtalen, kan 
    RF avbryte avtalen mellom Ares og RF.
\end{itemize}
\pagebreak
\setlength{\unitlength}{0.5mm}
\begin{picture}(300, 50)(-15,100)
  \put(0, 40){\text{Dato/sted:}}
  \put(10,10){\line(1,0){80}}
  \put(150,10){\line(1,0){80}}
  \put(150,0){\text{Utlånsansvarlig fra RF}}
  \put(10,0){\text{Ansvarlig fra Ares}}
\end{picture}

\end{document}
