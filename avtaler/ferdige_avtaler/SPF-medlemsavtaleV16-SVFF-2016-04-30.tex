\documentclass[12pt]{article}
\usepackage[utf8]{inputenc}
\usepackage[norsk]{babel}
\usepackage[usenames,dvipsnames]{xcolor}
\usepackage{lmodern}
\usepackage[a4paper, margin=0.75in]{geometry}
\usepackage{lastpage}
\usepackage{fancyhdr}
\usepackage{parskip}
\usepackage{titlesec} 
\usepackage[T1]{fontenc}
\usepackage[pdftex]{graphicx}
\usepackage[none]{hyphenat}
\usepackage{lastpage}

\titleformat{\subsection}[runin]
    {\normalfont\normalsize\bfseries}{\thesubsection}{1em}{}
\titleformat{\subsubsection}[runin]
    {\normalfont\normalsize\bfseries}{\thesubsubsection}{1em}{}

\begin{document}
\pagestyle{fancy}
\fancyhf{}
\fancyhead[L]{\textit{Medlemsavtale mellom SPF og Samfunnsvitenskapelig Fakultetsforening av \today}}
\fancyhead[R]{\thepage\ av\ \pageref{LastPage}}

\begin{center}
    {\LARGE\textbf{Medlemsavtale mellom Studentkjellernes personalforening
    (SPF) og Samfunnsvitenskapelig Fakultetsforening}}\\[7pt]
    av \today\\[24pt]
\end{center}

Formålet til Studentkjellernes
personalforening (heretter: «SPF») er å utføre 
personaltjenester for sine medlemsforeninger. Studentforeninger ved Universitetet i Oslo
(heretter: «UiO») som driver med utleievirksomhet.

Samfunnsvitenskapelig Fakultetsforening (heretter: «foreningen»)
er medlem av SPF, og driver kjelleren U1 (heretter: «kjelleren»). Rettigheter og plikter mellom 
foreningen og SPF er regulert av denne 
medlemsavtalen, av vedtektene til SPF og av punktene 13) og 
14) i \textit{Avtale om studentforeningers bruk og drift av 
studentkjellere ved Universitetet i Oslo
(«Kjelleravtalen»)} mellom foreningen og 
Universitetet i Oslo.
\section{Foreningens representasjon i SPF}
\label{sec:1}
\subsection{}
\label{sub:1.1}
Foreninger som er medlemmer av SPF skal ha en 
representant (med vara) i styret i SPF, jf. SPFs 
vedtekter 3.4. Denne representanten utpekes av 
foreningen.
\section{Ansvar for personalet}
\label{sec:2}
\subsection{}
\label{sub:2.1}
All betaling til personer som arbeider i foreningens kjeller / serveringslokaler
skal foregå gjennom SPF. Foreningen har ikke anledning til å utbetale lønn på egenhånd 
(jf. kjelleravtalen punkt 13).
\subsection{}
\label{sub:2.2}
SPF har arbeidsgiveransvar for alle personer som 
har lønnet arbeid gjennom foreningen. SPF tegner 
yrkesskadeforsikring for disse for det arbeidet de 
utfører som er lønnet gjennom SPF, men ikke for 
annet, frivillig, arbeid for foreningen (jf. 
kjelleravtalen punkt 14).
\subsection{}
\label{sub:2.3}
Foreningen bestemmer hvilke personer som til enhver 
tid skal arbeide i kjelleren.
\subsection{}
\label{sub:2.4}
SPF forutsetter at de personer som skal lønnes av 
SPF har opplæring i de oppgaver de skal utføre i 
henhold til de krav foreningen, UiO og myndighetene 
stiller. Det er foreningens ansvar å sørge for at 
personalet har denne opplæringen og den nødvendige 
sertifiseringen.
\section{Økonomi}
\label{sec:3}
\subsection{}
\label{sub:3.1}
Foreningen bestemmer selv satser for lønn for arbeid i sin kjeller.
\subsection{}
\label{sub:3.2}
SPF fakturerer foreningene for den lønnen som skal 
utbetales, med et påslag som skal dekke offentlige 
avgifter, forsikringer og administrative kostnader 
for SPF. Påslaget angis som en prosentsats av 
lønnen som skal utbetales. SPF har rett til å endre 
denne satsen, gitt at foreningen blir varslet 
skriftlig to måneder på forhånd. Størrelsen på 
påslaget er angitt i et vedlegg til denne avtalen. 
Innholdet i vedlegget kan endres uavhengig av denne 
avtalen.

\section{Registrering av person- og betalingsinformasjon. Frister.}
\label{sec:4}
\subsection{}
\label{sub:4.1}
Det er foreningens ansvar å innhente de opplysninger om personalet som SPF trenger for at lønnsutbetaling kan finne sted. SPF informerer foreningen om hvilke opplysninger dette er.
Opplysningene skal stilles til veie for SPF på den måten SPF spesifiserer.
\subsection{}
\label{sub:4.2}
SPF utbetaler lønn hver måned (unntatt juli) på en fast dato.
\subsection{}
\label{sub:4.3}
Foreningen oversender til SPF hver måned oversikt over hvor mye den enkelte skal ha 
utbetalt.
Dette må skje i den formen SPF spesifiserer
og innen angitte frister spesifisert i vedlegget.
SPF vil på en fast tid hver måned sende ut faktura
til foreningen for lønn som skal utbetales siden forrige utbetaling.

\subsection{}
\label{sub:4.5}
Fristene og datoene nevnt i avsnitt
\ref{sub:4.2}--\ref{sub:4.3} er angitt
i et vedlegg til denne avtalen.
Innholdet i vedlegget kan endres
uavhengig av denne avtalen.
\subsection{}
\label{sub:4.6}
SPF forplikter seg til å oppbevare
personopplysninger på en forsvarlig måte,
og til ikke å bruke dem i annen
sammenheng enn i forbindelse med utbetaling
av lønn og innrapportering til
myndighetene.
\subsection{}
\label{sub:4.7}
Foreningen har rett til å få oversikt
over registrert personale
og utbetalinger for sin forening.
\section{Andre ansvarsforhold}
\label{sec:5}
\subsection{}
\label{sub:5.1}
SPF har ingen skjenkerettigheter,
og står derfor ikke ansvarlig for disse.
\subsection{}
\label{sub:5.2}
SPF står ikke ansvarlig for hærverk
eller skader på UiOs eiedendom,
med mindre det direkte involverer
en av de ansatte i vedkommendes arbeidstid.
\subsection{}
\label{sub:5.3}
Foreningen plikter å gjøre denne avtalen 
kjent for nye styrer og ansvarspersoner i 
forbindelse med kjellerdriften.
\section{Endringer og oppsigelse}
\label{sec:6}
\subsection{}
\label{sub:6.1}
Denne avtalen løper til den blir sagt opp.
Endringer i den kan gjøres når som helst 
hvis begge partene er enige om det.
\subsection{}
\label{sub:6.2}
Hver av partene kan si opp avtalen med tre 
måneders skriftlig varsel.
UiO vil bli informert om dette, da det vil 
kunne få konsekvenser for foreningens 
kjelleravtale.
\subsection{}
\label{sub:6.3}
Hvis foreningen bryter avtalen (for 
eksempel ved å utbetale lønn selv eller 
ved ikke å overholde frister), kan SPF 
suspendere eller oppheve foreningens 
medlemskap i SPF.
UiO vil bli informert om dette, da det vil 
kunne få konsekvenser for foreningens 
kjelleravtale.
\subsection{}
\label{sub:6.4}
Utfyllende bestemmelser kan om nødvendig 
angis i egne vedlegg til denne avtalen.
Innholdet i vedlegget kan endres uavhengig
av denne avtalen.
Både SPF og foreningen må underskrive
vedlegg for at de skal være gyldige.

\setlength{\unitlength}{0.5mm}
\begin{picture}(300, 50)(30,100)
    \put(20, 50){Oslo, }
    \put(40,50){\line(1,0){50}}
    \put(30,10){\line(1,0){100}}
    \put(170,10){\line(1,0){100}}
    \put(170,0){For Studentkjellernes personalforening}
    \put(30,0){For foreningen}
\end{picture}
\end{document}
